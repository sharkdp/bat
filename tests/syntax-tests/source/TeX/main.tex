% !TeX program = lualatex
 
\documentclass{article}
\usepackage{verse}
\usepackage{fontspec}

\newcommand{\attrib}[1]{\nopagebreak{\raggedleft\footnotesize #1\par}}
\renewcommand{\poemtitlefont}{\normalfont\large\itshape\centering}

\setmainfont{DejaVu Sans}

\begin{document}

\poemtitle{სევდამოსილი}
\settowidth{\versewidth}{Than Tycho Brahe, or Erra Pater:}

\begin{verse}[\versewidth]
    ცა პირს შეიკრავს, ჩამობნელდება, \\
    გრილი ნიავი მოცერავს ფერდობს, \\
    ჩვენი სიცოცხლე უმალ ნელდება, \\
    იმედი აქცევს არსს უმოქმედოს. \\

    \bigskip

    სხვისი წესებით აგებულ სხეულს, \\
    მიჯაჭვულია გონებით, ხორცით, \\
    მილიონიდან განსახებს ეულს, \\
    დილის ნათება ეწყება ლოცვით. \\

    \bigskip

    ახალგაზრდაა, დიდსულოვანი, \\
    ზოგჯერ რაინდი, ხანაც მგოსანი, \\
    თავდადებული, კონტრნაღმოსანი, \\
    თვისი სამიზნე - დოტას როშანი. \\

    \bigskip

    თუ შეიყვარებ, არასდროს გავნებს, \\
    დღეგამოშვებით იბარებს თავნებს, \\
    ყველას ჰპატიობს, გულქვას და თავნებს, \\
    სხვის მაგივრადაც საკუთარ თავს ვნებს. \\

    \bigskip

    მისი სახელი - თენგიზი (დიდი), \\
    სახის იერი - ნაზი და მშვიდი, \\
    ვიზუალურად - ათიდან შვიდი, \\
    ხანმოკლე ვითარც ეფემერიდი. \\

    \bigskip

    მე ის მახარებს რომ სხვას ახარებ, \\
    შაირს რომ იტყვი, იმასხარავებ, \\
    კეთილ საქმეს რომ არვის ახარბებ, \\
    გულიანად რომ გადიხარხარებ. \\

    \bigskip

    როცა დაგჭირდეს, ჭირში თუ ლხინში, \\
    მიწაზე, წყალში... ნავში თუ ქარში, \\
    ჩათვალე, მიდგას კოდურად ჯინში, \\
    სათქმელი არის? იქნება პირში. \\

    \bigskip

    შენ არ იჯავრო, - კარგად იქნები, \\
    მიხედე შენს თავს, ეძიე უკვლევს, \\
    მეგზურად გყვება ჩუმი ფიქრები, \\
    შეუმჩნევლად რომ გითვლიან სულ წლებს. \\

    \bigskip

    გილოცავ ამ დღეს, დედამ რომ გშობა, \\
    როცა შეჰმატე გარემოს ფერები, \\
    ერთად გეტაროთ აღდგომა, შობა, \\
    მე კი ლექსიდან მოგეფერები. \\

    \bigskip
\end{verse}

\attrib{გიორგი ბერიაშვილი (1999--$\infty)$}

\end{document}
